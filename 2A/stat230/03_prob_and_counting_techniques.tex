\begin{remark}
    $\displaystyle P(A) = \frac{\text{\# of oucomes in $A$}}{\text{\# of outcomes in $S$}}$
\end{remark}

\subsection{Addition and Multiplication Rules}

\begin{definition}[\textbf{Counting Rules}]
    \phantom{}
    \begin{enumerate}
        \item \textbf{Addition Rule:} Suppose we can do job $1$ in $p$ ways and job 2 in $q$ ways.
        Then we can do either job $1$ \textbf{OR} job $2$ (but not both), in $p + q$ ways.
        \item \textbf{Multiplication Rule:} Suppose we can do job $1$ in $p$ ways and,
        for each of these ways, we can do job $2$ in $q$ ways. Then we can do both
        job $1$ \textbf{AND} job $2$ in $p \times q$ ways.
    \end{enumerate}
\end{definition}

\begin{note}
    \phantom{}
    \begin{itemize}
        \item "OR": interpreted as \textbf{addition}.
        \item "AND": interpreted as \textbf{multiplication}.
        \item "With" replacement: every time an object is selected, we put it back into the pool of possible objects (could be picked again).
        \item "Without" replacement: every time an object is selected, it is NOT put back.
    \end{itemize}
\end{note}

\begin{example}
    A bag contains 3 blue marbles and 5 red marbles. Find the probability of selecting two blue marbles \textbf{with and without} replacement.  \\
    \textbf{Solution: } With replacement, $P(BB) = P(B \text{ AND } B) = \frac{3}{8} \times \frac{3}{8} = \frac{9}{64}$.    \\
    \phantom{\textbf{Solution : }}Without replacement, $P(BB) = P(B \text{ AND } B) = \frac{3}{8} \times \frac{2}{7} = \frac{3}{28}$.
\end{example}


\subsection{Counting Arrangements or Permutations}

\begin{definition}[Permutation]
    A \textbf{permutation} is an arrangement of objects in a definite order (order matters and selection is done without replacement).
\end{definition}

\begin{example}
    How many different ordered arrangements of the letters a, b, and c are possible.    \\
    \textbf{Solution: } $\underbrace{\text{\underline{3} \underline{2} \underline{1}}}_{\text{letters available for each selection}}$
    So, $3 \times 2 \times 1 = 6$ ways. \\
\end{example}

To generalize, in each case we count the \# of arrangements by counting the \# of ways we can 
fill the positions in the arrangement. Suppose we have $n$ symbols, we can make:
\begin{itemize}
    \item $n! = n \times (n-1) \times \cdots \times 1$ arrangements of length $n$ using each symbol once and only once
    (the \# of permutations of $n$ distinct objects).
    \item $n^{(k)} = n \times (n-1) \times \cdots \times (n-k+1)$ arrangements of length $k$ using each symbol at most once.
    Note that $n^{(k)} = \frac{n!}{(n-k)!} = \prescript{n}{}{P}_{k}$ (read "$n$ to $k$ factors").
    \item $n^k = n \times n \times \cdots \times n$ arrangements of length $k$ using each symbol more than once.    \\
\end{itemize}

\begin{theorem}[\textbf{Stirling's Approximation}]
    \phantom{}  \\
    For large $n$ there is an approximation to $n!$, it states that $n!$ is \textbf{asymptotically equivalent} to $\dis (\frac{n}{e})^n \sqrt{2 \pi n}$.
    Note that the sequence $\{a_n\}$ is asymptotically equivalent to the sequence $\{b_n\}$ if $\dis \lim_{n \to \infty}\frac{a_n}{b_n} = 1$.
\end{theorem}

\begin{note}
    This implies that as $n \to \infty$, the approximation becomes better and better    \\
    (i.e. $\dis \lim_{n \to \infty}\frac{(\frac{n}{2})^n \sqrt{2 \pi n}}{n!} = 1$). \\
\end{note}

\begin{example}
    A pin number of length four is formed by randomly selecting four digits from the set $\{0, 1, 2, \ldots, 9\}$ \textbf{with replacement}.
    Find the probability of the events:
    \begin{itemize}
        \item $A$: the pin number is even.
        \item $B$: the pin number contains at least one 1.
    \end{itemize}
    \textbf{Solution: }
    \begin{enumerate}[label={(\alph*)}]
        \item $P(A) = \frac{\text{\# of pins that are even}}{\text{\# of pins in the sample space}} = \frac{10 \times 10 \times 10 \times 5}{10 \times 10 \times 10 \times 10} = \frac{1}{2}$.
        \item We can use the complement. Find the pin numbers that do not contain 1. \\ 
        $P(B) = 1 - P(\overline{B}) = 1 - \frac{9^4}{10^4} = 0.3439$.
    \end{enumerate}
    Exercise: try this \textbf{without replacement}.
\end{example}

\newpage

\begin{example}
    Five separate awards (best scholarship, best leadership qualities, and so on) are to be 
    presented to selected students from a class of 30. How many different outcomes are possible if:

    \begin{enumerate}[label={(\alph*)}]
        \item A student can receive any number of awards?   \\
        \textbf{Solution: } $30^5$.
        \item Each student can receive at most 1 award?     \\
        \textbf{Solution: } $30 \times 29 \times 28 \times 27 \times 26 = 30^{(5)}$.
    \end{enumerate}
\end{example}

\begin{example}
    \phantom{}
    \begin{enumerate}[label={(\alph*)}]
        \item In how many ways can 3 boys and 3 girls sit in a row? \\
        \textbf{Solution: } $6! = 120$.
        \item In how many ways can 3 boys and 3 girls sit in a row if the boys and the girls are each to sit together? \\
        \textbf{Solution: } $2! \times 3! \times 3! = 72$.
        \item In how many ways if only the boys must sit together? \\
        \textbf{Solution: } $4! \times 3! = 144$.
        \item In how many ways if no two people of the same sex are allowed to sit together? \\
        \textbf{Solution: } $3! \times 3! \times 2! = 72$.
    \end{enumerate}
    
\end{example}


\subsection{Counting Subsets or Combinations}

\begin{definition}[Combination]
    A combination is an unordered selection of $k$ objects chosen from $n$ objects.
\end{definition}

\begin{remark}[\textbf{Number of subsets of size $k$}]
    We use $\displaystyle \prescript{n}{}{C}_{k} = \binom{n}{k}$ to
    denote the \# of subsets of size $k$ that can be selected from a set of $n$ objects.
    Then, $m \times k! = n^{(k)}$ and we have
    \[\prescript{n}{}{C}_{k} = \binom{n}{k} = \frac{n^{(k)}}{k!} = \frac{n!}{k!(n-k)!}.\]
\end{remark}

\textbf{Properties of $\binom{n}{k}$: }
\begin{enumerate}
    \item $\binom{n}{k} = \frac{n^{(k)}}{k!} = \frac{n!}{k!(n-k)!} = \frac{n(n-1)\cdots(n-k+1)}{k!}$ if $n \in \R$ and $k$ is a non-negative integer
    \item $\binom{n}{k} = \binom{n}{n-k}$ for all $k = 0, 1, \ldots, n$
    \item $\binom{n}{0} = \binom{n}{n} = 1$
    \item $\binom{n}{k} = \binom{n-1}{k-1} + \binom{n-1}{k}$
    \item \textbf{Binomial Theorem: } $(1 + x)^n = \binom{n}{0} + \binom{n}{1}x + \binom{n}{2}x^2 + \cdots + \binom{n}{n}x^n$
\end{enumerate}


\subsection{Number of Arrangements when Symbols are Repeated}

\begin{example}
    Suppose the letters of the word "STATISTICS" are arranged at random. Find the 
    probability of the event G that the arrangement begins and ends with "S". \\
    \textbf{Solution: } $P(G) = \frac{\text{\# of subsets in G}}{\text{\# of subsets in the sample space}}$. \vspace{1mm} \\
    The number of equally probable outcomes in the sample space $S$ is:
    \[
        \underbrace{\binom{10}{3}}_{\text{S}} \underbrace{\binom{7}{3}}_{\text{T}} \underbrace{\binom{4}{2}}_{\text{I}}
        \underbrace{\binom{2}{1}}_{\text{C}} \underbrace{\binom{1}{1}}_{\text{A}} =
        \frac{10!}{3!7!} \frac{7!}{3!4!} \frac{4!}{2!2!} \frac{2!}{1!1!} \frac{1!}{1!0!} = \frac{10!}{3!3!2!1!1!}.
    \]
    The number of arrangements in $G$:
    \[
        \underbrace{\binom{8}{1}}_{\text{S}} \underbrace{\binom{7}{3}}_{\text{T}} \underbrace{\binom{4}{2}}_{\text{I}}
        \underbrace{\binom{2}{1}}_{\text{C}} \underbrace{\binom{1}{1}}_{\text{A}} = \frac{8!}{1!3!2!1!1!}.
    \]

    Thus, $P(G) = \dis \frac{\frac{8!}{1!3!2!1!1!}}{\frac{10!}{3!3!2!1!1!}} = \frac{1}{15}$. \\
\end{example}

\begin{remark}[\textbf{Number of arrangements when symbols are repeated}]
    \phantom{}  \\
    If we have $n_i$ symbols of type $i$, $i = 1, 2, \ldots, k$ with $n_1 + n_2 + \cdots + n_k = n$,
    then the \# of arrangements using all of the symbols is
    \[\displaystyle \binom{n}{n_1}\binom{n - n_1}{n_2} \binom{n - n_1 - n_2}{n_3} 
    \cdots \binom{n_k}{n_k} = \frac{n!}{n_1! n_2! \cdots n_k!}.
    \] \\
\end{remark}

\begin{example}
    Find the probability a bridge hand (13 cards picked at random from a standard deck of 52 cards without replacement) has
    \begin{enumerate}[label={(\alph*)}]
        \item at least 1 Ace.
        \item 6 spades, 4 hearts, 2 diamonds and 1 club.
        \item a 6-4-2-1 spilt between the 4 suits.
    \end{enumerate}
    \textbf{Solution (a): } \vspace*{-5mm}
    \begin{align*}
        P(\text{at least 1 Ace}) &= 1 - P(\text{0 Aces}) \\
        &= 1 - \frac{\binom{4}{0} \binom{52-4}{13}}{\binom{52}{13}}
    \end{align*}

    \textbf{Solution (b): } \vspace*{-3mm}
    $\frac{\binom{13}{6} \binom{13}{4} \binom{13}{2} \binom{13}{1}}{\binom{52}{13}}$. \\

    \textbf{Solution (c): } \vspace*{-3mm}
    $\frac{4! \cdot \binom{13}{6}\binom{13}{4}\binom{13}{2}\binom{13}{1}}{\binom{52}{13}}$. \vspace{2mm}
\end{example}

\begin{example}
    If 12 people are to be divided into 3 committees of respective sizes 3, 4, and 5,
    how many divisions are possible? \\
    \textbf{Solution: }
    $\binom{12}{3} \binom{9}{4} \binom{5}{5} = 27720$.
\end{example}

\begin{example}
    A person has 8 friends, of whom 5 will be invited to a party. 
    \begin{enumerate}[label=(\alph*)]
        \item How many choices are there if 2 friends are feuding and will not attend together? \\
        \textbf{Solution: } 
        $\binom{8}{5} - \binom{2}{2}\binom{6}{3} = 36$ ways (sample space - both feuding friends come).
        \item How many choices are there if 2 of the friends will only attend together? \\
        \textbf{Solution: }
        $\underbrace{\binom{2}{2} \binom{6}{3}}_{\text{Both are invited}}  + \underbrace{\binom{2}{0} \binom{6}{5}}_{\text{Both don't go}}  = 20 + 6 = 26$. 
    \end{enumerate}
\end{example}

\begin{example}
    There are 5 blue beads and 4 green beads to be arranged in a row on a string. The two ends of
    the string are not connected. Beads with the same colour are indistinguishable. Find the probability of
    the following events:
    \begin{enumerate}[label=(\alph*)]
        \item A = “All 5 blue beads are adjacent to each other”. \vspace{1mm} \\
        \textbf{Solution: } $P(A) = \frac{\binom{5}{1} \binom{4}{4}}{\binom{9}{5} \binom{4}{4}} = \frac{\frac{5!}{1!4!}}{\frac{9!}{1!4!}}$.
        \item B = “None of the green beads is adjacent to any other green beads”. \vspace{1mm} \\
        \textbf{Solution: } $P(B) = \frac{\binom{6}{4}}{\binom{9}{5} \binom{4}{4}}$.
    \end{enumerate}
    
\end{example}

\newpage