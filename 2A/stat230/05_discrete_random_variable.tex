\subsection{Random Variables and Probability Functions}

\begin{definition}[\textbf{Random Variable}]
    A \textbf{random variable} is a function that assigns a real number to each point
    in a sample space $S$.
\end{definition}

\begin{example}
    Suppose an experiment consists of tossing a coin 3 times. The sample space
    \[S = \left\{ \text{HHH, THH, HTH, HHT, HTT, THT, TTH, TTT} \right\}.\]
    Then $X = $ number of heads that occur would be a random variable, associated with range $A=\left\{ 0,1,2,3 \right\}$. \vspace{-5mm} \\
    \begin{center}
        \begin{tabular}{l|*{1}{c}}
            Events & Outcomes from $S$ \\
            \hline
            $X=0$ & $\left\{ \text{TTT} \right\}$ \\
            $X=1$ & $\left\{ \text{HTT, THT, TTH} \right\}$ \\
            $X=2$ & $\left\{ \text{HHT, HTH, THH} \right\}$ \\
            $X=3$ & $\left\{ \text{HHH} \right\}$
        \end{tabular}
    \end{center}
    Ex. $P(X=1) = P(\text{HTT $\cup$ THT $\cup$ TTH}) = P(\text{HTT}) + P(\text{THT}) + P(\text{TTH}) = \frac{3}{8}$.
\end{example}

\begin{definition}[\textbf{Discrete Random Variables}]
    \textbf{Discrete random variables} take integer values or, more generally, values
    in a countable set.
\end{definition}

\begin{note}
    A set is countable if its elements can be placed in a one-to-one correspondence
    with a subset of the positive integers.
\end{note}

\begin{definition}[\textbf{Continuous Random Variables}]
    \textbf{Continous random variables} take values in some interval of real numbers like
    $(0, 1)$ or $(0, \infty)$ or $(-\infty, \infty)$. 
\end{definition}

\begin{note}
    Cardinality of the real numbers in an interval is NOT countable.
\end{note}

\begin{definition}[\textbf{Probability Function \& Probability Distribution}]
    \phantom{}  \\
    Let $X$ be a discrete random variable with
    $\operatorname{range}{(X)} = A$. The \textbf{probability function} of $X$ is the function
    \[f(x) = P(X = x), \quad \text{defined $\forall \, x \in A$.}\]

    The set of pairs $\{(x, f(x)) : x \in A\}$ is called the \textbf{probability distribution} of $X$.  \\
    All probability functions must have two properties:
    \begin{enumerate}
        \item $f(x) \geq 0 \; \forall x \in A$.
        \item $\dis \sum_{\text{all } x \in A} f(x) =1$.
    \end{enumerate}
\end{definition}

\begin{remark}
    It follows that $0 \leq f(x) \leq 1 \; \forall x \in A$.
\end{remark}

\begin{example}
    The random variable $X$ has probability function given by \vspace{-5mm} \\
    \begin{center}
        \begin{tabular}{l|*{5}{c}}
            $x$ & 0 & 1 & 2 & 3 & 4 \\
            \hline
            $f(x)$ & $0.1c$ & $0.2c$ & $0.5c$ & $c$ & $0.2c$ \\
        \end{tabular}
    \end{center}
    
    \begin{enumerate}[label=(\alph*)]
        \item Find $c$. \\
        \textbf{Solution:} Since $\dis \sum_{\text{all } x \in A} f(x) =1$. Then $0.1c + 0.2c + 0.5c + c + 0.2c = 1$.
        This gives $c = 0.5$.
        \item Find $P(X > 2)$. \\
        \textbf{Solution:} $P(X > 2) = P(X \geq 3) = P(X = 3) + P(X = 4) = c + 0.2c = 1.2c = 0.6$.
    \end{enumerate}
\end{example}

\begin{definition}[\textbf{Cumulative Distribution Function}]
    The \textbf{cumulative distribution function} of the discrete random variable $X$
    is the function usually denoted by $F(X)$.
    \[F(x) = P(X \leq x) \quad \text{defined $\forall \, x \in \R$}\]
\end{definition}

In general, $F(x)$ can be obtained from $f(x)$ using:
\[F(x) = P(X \leq x) = \displaystyle \sum_{u\leq x} f(u).\]

\vspace{-5mm}

\begin{theorem}[\textbf{Properties of a CDF $F(x)$}]
    \phantom{}
    \begin{enumerate}
        \item $F(x)$ is a non-decreasing function of $x \; \forall x \in \R$. For example, $P(X \leq 8)$ cannot be less than $P(X \leq 7)$.
        \item $0 \leq F(x) \leq 1 \; \forall x \in \R$.
        \item $\dis \lim_{x \to -\infty} F(x) = 0$ and $\dis \lim_{x \to \infty} F(x) = 1$.
    \end{enumerate}
\end{theorem}

We can also obtain $f(x)$ from $F(x)$. If $X$ takes on integer values then for values $x$ \st $x \in A$ and $x-1 \in A$, 
\[f(x) = F(x) - F(x-1).\]
In other words:
\[P(X = x) = P(X \leq x) - P(X \leq x-1).\]
We notice that $f(x)$ represents the size of the jump in $F(x)$ at the point $x$.


\textbf{Plots of $f(x)$ and $F(x)$} \\
For discrete random variables, the cdf $F(x)$ is represented as a step function, whereas the pf $f(x)$ is represented by a histogram.


\subsection{Discrete Uniform Distribution}

\textbf{Physical Setup:}  Suppose the range of $X$ is $\{a, a + 1, \ldots, b\}$ where $a$ and $b$ are integers and suppose all values are equally probable.
Then $X$ has a Discrete Uniform distribution on the set $\{a, a + 1, \ldots, b\}$. The variables $a$ and $b$ are called the parameters of the distribution.

\textbf{Probability Function:} There are \(b - a + 1\) values in the set \(\{a, a + 1, \ldots, b\}\) so the probability of each value must be \(\frac{1}{b - a + 1}\) in order that \(\dis \sum_{x=a}^{b} f(x) = 1\). Therefore

\[
    f(x) = P(X = x) = 
    \begin{cases} 
        \frac{1}{b - a + 1} & \text{for } x = a, a + 1, \ldots, b \\
        0 & \text{otherwise}
    \end{cases}
\]

\begin{example}
    Suppose a fair die is thrown once and let $X$ be the number on the face. Find the cumulative distribution function of $X$. \\
    \textbf{Solution:} $x = 1,2,3,4,5,6$ and $X \sim \text{Unif}[1,6]$.
    \[
        f(x) = P(X = x) = 
        \begin{cases} 
            \frac{1}{6} & \text{for } x = 1,2,\ldots,6 \\
            0 & \text{otherwise}
        \end{cases}
    \]
    $F(1) = P(X \leq 1) = P(X = 1) = \frac{1}{6}$ and $F(2) = P(X \leq 2) = P(X = 1) + P(X = 2) = \frac{2}{6}$. So
    \[
        F(x) = P(X \leq x) = 
        \begin{cases} 
            0 & \text{for } x < 1 \\
            \frac{[x]}{6} & \text{for } 1 \leq x < 6 \\
            1 & \text{for } x \geq 6 
        \end{cases}.
    \]
\end{example}


\subsection{Hypergeometric Distribution}

\textbf{Physical Setup:}
We have \textbf{\textit{N objects}} that can be classified into exactly \textbf{\textit{two}} distinct types, “success” (S) vs. “failure” (F).

$r =$ \# of successes \quad and \quad $N-r=$ \# of failures.

We pick \underline{$n$} objects at random \textbf{without} replacement. If X represents the number of successes, then X has a hypergeometric distribution.

\begin{example}
    Of the 120 applicants for a job, only 80 are qualified. In total, 5 applicants are picked at random for an interview. If X represents the number of qualified
    applicants that are interviewed, then X has a hypergeometric distribution, where:
    \begin{itemize}
        \item $N = 120$
        \item $r = 80$ since $N - r = 40$
        \item $n = 5$
        \item Possible values of $x$ are 0, 1, 2, 3, 4, 5.
    \end{itemize}
\end{example}

\textbf{Probability Function:} Using counting techniques we note there are \(\binom{N}{n}\) points in the sample space
\(S\) if we don't consider order of selection. There are \(\binom{r}{x}\) ways to choose the \(x\) success objects from the
\(r\) available and \(\binom{N-r}{n-x}\) ways to choose the remaining \((n - x)\) objects from the \((N - r)\) failures. Hence

\[
    f(x) = P(X = x) = \frac{\binom{r}{x} \binom{N-r}{n-x}}{\binom{N}{n}}
\]
where $x \geq \operatorname{max}(0,n-(N-r))$ and $x \leq \operatorname{min}(r,n)$.

\begin{note}
    Do not need to memorize this formula because it will be easy to realize based on the given context.
\end{note}

\begin{example}[\textbf{Above Continued}]
    \phantom{}\\
    Find the probability that only two of the five selected will be qualified for the job. \vspace{2mm} \\ 
    \textbf{Solution:} $P(X = 2) = \dis \frac{\binom{80}{2} \binom{40}{3}}{\binom{120}{5}} = 0.164$.
\end{example}

\begin{example}
    In the game of Texas Hold'em, players are each dealt two private cards, and five community cards are dealt face-up on the table.
    Each player makes the best 5-card hand they can with their two private cards and the five community cards.
    What is the probability that a particular player can make a flush of spades (i.e. 5 spades or more)?

    \textbf{Solution:} $X = \#$ of spades among 7 cards. $N = 52$, $r = 13$, $N-r = 39$, $n = 7$, and $x = 0,1,2,\ldots ,7$. \vspace{-4mm}
    \begin{align*}
        P(X \geq 5) &= P(X=5)+ P(X=6) + P(X=7) \\
        &= \dis \frac{\binom{13}{5}\binom{39}{2} + \binom{13}{6}\binom{39}{1} + \binom{13}{7}\binom{39}{0}}{\binom{52}{7}} \\
        &= 0.0076.
    \end{align*}
\end{example}

\begin{example}
    A manufacturer of auto parts just shipped 25 auto parts to a dealer. Later, it found out that 5 of those parts were defective. By the time the company
    manager contacted the dealer, 4 auto parts from that shipment had been sold. What is the probability that 3 of those 4 parts were good parts and one was defective?

    \textbf{Solution:} $X = \#$ of defectives selected, $N = 25$, $r = 5$, and $n = 4$. \vspace{-3mm}
    \[
        P(X = 1) = \dis \frac{\binom{5}{1}\binom{25-5}{4-1}}{\binom{25}{4}} = 0.45.
    \]
\end{example}


\subsection{Binomial Distribution}

\textbf{Physical Setup:} Suppose an experiment has \textbf{\textit{two}} possible outcomes, ``success'' and ``failure''.
Let $P(\text{Success}) = p$ and hence $P(\text{Failure}) = 1-p$. Repeat the experiment \textbf{\textit{$n$ independent}} times.
Let $X$ be the number of successes obtained, we say $X$ has a \textbf{\textit{Binomial distribution}}.


We write \( X \sim Bin(n, p) \) with $n$ total trials and $p$ probability to success.

\begin{note}
    The \textbf{\textit{n individual}} experiments are called “trials” or “Bernoulli trials” and the process is
    called a Bernoulli process or Binomial process. \\
\end{note}


Underlying assumptions for the Binomial Distribution: (Tims)
\begin{itemize}
    \item T: Two outcomes
    \item I: Independent trials
    \item M: Multiple trials
    \item S: Same probability of success in each trial.
\end{itemize}

\begin{example}
    A fair coin is tossed 12 times. Let X be the number of heads obtained. Then:
    \begin{itemize}
        \item Two outcomes: Head (success) vs. Tail (failure)
        \item Independent trials: flips are independent of each other
        \item Multiple trials: $n = 12$
        \item Same $P(\text{Success})$ in each trial = $p$ = 0.5.
    \end{itemize}
    Thus, $X \sim Bin(12, 0.5)$ and $x = 0,1,2, \ldots ,12$. \\
\end{example}

\textbf{Probability Function:} There are $\binom{n}{x}$ arrangements of $x$ successes and $(n-x)$ failures over the $n$ trials.
Each arrangement has probability $P^x(1-p)^{n-x}$. Therefore,
\[f(x) = P(X=x) = \binom{n}{x}p^x(1-p)^{n-x} \quad \text{for $x = 0,1,\ldots ,n$ and $0 < p < 1$}.\]

\begin{remark}
    Check the above function by using the property that $\displaystyle \sum_{\text{all $x$}} f(x) = 1$ for $0 < p < 1$.
\end{remark}

\begin{example}
    Seventy-five percent of students at a college with a large student population use Instagram.
    A sample of five students from this college is selected. What is the probability that at least 3 use Instagram?

    \textbf{Solution:} $X = \#$ of students that use Instagram among 5 selected. Then $X \sim Bin(5,0.75)$ for $x=0,1,\ldots ,5$.
    \[P(X \geq 3) = P(X=3) + P(X=4) + P(X=5) = 1- P(X \leq 2).\]
\end{example}

\begin{remark}
    \phantom{}\\
    The probability of at least `x': $P(X \geq x) = 1 - P(X \leq (x-1))$. \\
    The probability of more than `x': $P(X > x) = 1 - P(X \leq x)$.
\end{remark}

\textbf{Binomial vs. Hypergeometric}

Similarities
\begin{itemize}
    \item Both have two types of outcomes, success and failure.
    \item The experiment is repeated $n$ times.
    \item $X$ records the number of successes.
\end{itemize}

Differences
\begin{itemize}
    \item Binomial reqires \textbf{independent} trials, where the probability of success is the same in each trial.
    \item In Hypergeometric, the draws are made from a fixed number of objects $N$ \textbf{without} replacement. Hence, the trials are \textbf{not independent}. \\
\end{itemize}

\begin{example}
    Suppose we have 20 cans of drinks placed in a big ice container such that the labels are not visible. It is known that 12
    are coke and 8 are juice. We randomly pick 10 cans. Find the probability that 3 are coke.

    \textbf{Solution:} Assume the draws are down without replacement (correct approach here). We have $N=20$, $r=12$, $N-r=8$, and $n=10$. So $X \sim \text{Hypergeo}(20, 12, 10)$.
    
    Thus, $P(X=3) = \frac{\binom{12}{3}\binom{8}{7}}{\binom{20}{10}} = 0.00953.$
\end{example}

\begin{remark}
    If $N$ is large and $n$ (number of objects being drawn) is relatively small, then binomial can be used as an approximation for the hypergeometric. \\

    \textbf{Rule of thumb:} If the sample size (number of trials) $n$ is at most $5\%$ of the population size, the experiment can be analyzed as though it were exactly a Binomial experiment.
\end{remark}

\begin{example}
    Megan audits 130 clients during a year and finds irregularities for 26 of them.
    \begin{enumerate}[label=(\alph*)]
        \item Give an expression for the probability that 2 clients will have irregularities when 6 of her
        clients are picked at random. \\
        \textbf{Solution:} Let $X = \#$ of clients with irregularities among 6 selected, $N=130$, $r=26$, $n=6$.
        \[P(X=2) = \frac{\binom{26}{2}\binom{130-26}{6-2}}{\binom{130}{6}} = 0.251.\]
        \item Evaluate your answer to (a) using a suitable approximation. \\
        \textbf{Solution:} $X \sim \text{Bin}(6,\frac{26}{130})$. Then we have
        \[P(X=2)=\binom{6}{2}\left(\frac{26}{130}\right)^2 \left(1-\frac{26}{130}\right)^{6-2} = 0.246 \approx 0.25.\]
        Notice that $\frac{n}{N} = \frac{6}{130} = 0.046 < 5\%$.
    \end{enumerate}
\end{example}


\subsection{Negative Binomial Distribution}

\textbf{Physical Setup:} similar to the Binomial Distribution. Do experiment until we obtain $k$ successes. However now, $X$ records the number of
failures obtained before the $k$th success, $X \sim \text{NB}(k,p)$.

\begin{example}
    Draw cards with replacement until you get 3 Aces. Let $X = \#$ of Non-Aces that we obtain before the third Ace. Then the distribution of $X$ is
    $X \sim \text{NB}(k=3, p=P(\text{Success})=\frac{4}{52})$ with $x =  0,1, \ldots$.
\end{example}


\textbf{Probability Function:} In total, we have $x$ failures and $k$ successes, so there are $x+k$ trials, the last trial \textbf{MUST} be a success.
Hence in the first $x+k-1$ trials, we observe $x$ failures and $(k-1)$ successes. Therefore, there are $\binom{x+k-1}{x} = \binom{x+k-1}{k-1}$ arrangements, each arrangement has probability $p^k(1-p)^x$.
Hence \[f(x) = P(X=x) = \binom{x+k-1}{x}p^k(1-p)^x \quad \text{for $x = 0,1,\ldots$ and $0 < p < 1$}.\]

\begin{note}
    An alternate version of Negative Binomial Distribution defined $X$ to be the total number of trials needed to get the $k$th success. We'll have something like: $X-4 \sim \text{NB}(k,p)$, 
    where 4 is just an example here, representing the number of successes before the 5th success.
\end{note}

\textbf{Binomial vs. Negative Binomial}
\begin{itemize}
    \item \textbf{Binomial:} We know that we have $n$ trials in advance but do not
    know the \# of successes we will obtain until after the experiment.
    \item \textbf{Negative Binomial:} We know the number $k$ of successes in advance but do
    not know the \# of trials that will be needed to obtain this \# of success until after the experiment.
\end{itemize}


\pagebreak

\begin{example}
    Two athletic teams, $A$ and $B$, play a best-of-three series of games (i.e the first team to win two games is the overall winner). Suppose team $A$ is the
    stronger team and will win any game with probability 0.6, independently from other games. Let $X$ be the number of games lost before Team $A$ wins twice.
    Find the probability that Team $A$ is the overall winner.

    \textbf{Solution:} $X \sim \text{NB}(2, 0.6)$. \vspace{-5mm}
    \begin{align*}
        \text{Probability} &= P(X = 0) + P(X = 1) \\
        &= \binom{0+2-1}{0}(0.6)^2(0.4)^0 + \binom{1+2-1}{1}(0.6)^2(0.4)^1
    \end{align*}
\end{example}

\begin{example}
    A startup is looking for 5 investors. Each investor will independently agree with probability 20\%.
    A founder asks investors one at a time until they get 5 ``yes''.
    Let $X$ be the total $\#$ of investors asked. Find $f(x)$ and $f(6)$.

    \textbf{Solution:} Let $Y = \#$ of ``No''s until the 5th yes is obtained. Then $Y \sim \text{NB}(5,0.2)$, $y = 0,1,2,\ldots$.
    And $X = Y + 5$. \vspace{-5mm}
    \begin{align*}
        f_X(x) &= P(X=x) \\
        &= P(X = x = Y + 5) \\
        &= P(Y = x - 5) \\
        &= f_Y(x-5) \\
        &= \binom{x-5+5-1}{x-5}p^5(1-p)^{x-5} \\
        &= \binom{x-1}{x-5}p^5(1-p)^{x-5} \text{, $x = 5,6,7,\ldots$}
    \end{align*}

    $\dis f_x(6) = P(X = 6) = \binom{6-1}{6-5}0.2^5(1-0.2)^{6-5} = 0.00128$.
    
\end{example}



\subsection{Geometric Distribution}

\textbf{Physical Setup:} Independent Bernoulli trials, each having two possible outcomes (Success vs. Failure). 
The probability, $p$, of success is the same each time. However, $X = \#$ of failures before the \textbf{FIRST} success
(i.e. a Negative Binomial distribution with $k=1$). 

We write $X \sim \text{Geo}(p)$.


\textbf{Probability Function:} There is only the one way with $x$ failures followed by 1 success.
\[f(x) = P(X=x) = (1-p)^x p \quad \text{for $x = 0,1,\ldots$ and $0<p<1$}.\]

\begin{remark}
    Alternatively, substitute $k=1$ in $f(x)$ for the Negative Binomial. \\
\end{remark}


In summary, we notice that Binomial, Negative Binomial and Geometric all assume:
\begin{enumerate}
    \item Two outcomes in each trial.
    \item Independent Trails.
    \item Each trial has the same probability of success. \\
\end{enumerate}



\begin{example}
    A company receives $60\%$ of its orders over the internet.
    \begin{enumerate}[label=(\alph*)]
        \item What is the probability that the fifth order received is the first internet order? \\
        \textbf{Solution:} Let $X = \#$ of orders not over the internet untill the first internet order. Then $X \sim \text{Geo}(0.6)$, $x=0,1,2,\ldots$.
        We have $P(X = 4) = (0.6)^1 (1-0.6)^4$.
        \item What is the probability that the eighth order received is the fourth internet order? \\
        \textbf{Solution:} Let $Y =$ total number of orders before the 4th internet order. $Y \sim \text{NB}(4, 0.6)$. Then $\dis f_y(4) = P(Y = 4) = \binom{4+4-1}{4}(0.6)^4(1-0.6)^4$. Since we have 4 successes and 4 failures.
        
        Alternatively, $Y =$ number of non-internet orders before 4th internet. Then $Y-3 \sim \text{NB}(4,0.6)$.
    \end{enumerate}
\end{example}


\subsection{Poisson Distribution from Binomial}

\textbf{Physical Setup:} $X = \#$ of events of some type. The events occur according to some rate $\mu$, where $\mu > 0$.
We write $X \sim \text{Poisson}(\mu)$.

\textbf{Probability Function:} 
\[
    f(x) = P(X = x) = \dis \frac{e^{-\mu} \mu^x}{x!} \quad \text{for } x = 0, 1, \ldots
\]
where $\mu > 0$.

\begin{remark}
    Poisson arises from Binomial when $n \to \infty$ and $p \to 0$, $X \sim \text{Bin}(n, \frac{\mu}{n})$. $(\mu = np)$
\end{remark}


\begin{example}
    Consider the Tim Hortons event ``Roll up the Rim''. We are told that 1 in 9 cups are winners.
    Say you buy 100 cups, assuming they are independent, and use the Poisson approximation to solve for the probability that you get no more than 10 winning cups.

    \textbf{Solution:} Find the exact probability using Binomial.

    Let $X = \#$ of winning cups among 100. Then $X \sim \text{Bin}(100, \frac{1}{9})$. We have $P(X \leq 10) = 0.439$.

    Now, use Poisson approximation. We have $\mu = np = 100 \times \frac{1}{9} = \frac{100}{9}$. Then

    $P(X \leq 10) = P(X = 0) + P(X = 1) + \cdots + P(X = 10) = e^{-\frac{100}{9}}\left[ \frac{\mu^0}{0!} + \frac{\mu^1}{1!} + \cdots + \frac{\mu^{10}}{10!}\right] = 0.447$. \\
\end{example}

\begin{example}
    If you buy a lottery ticket in 50 lotteries, in each of which your chance of winning a prize is
    1/100, what is the (approximate) probability that you will win a prize
    \begin{enumerate}[label=(\alph*)]
        \item At least once,
        \item Exactly once,
        \item At least twice?
    \end{enumerate}

    \textbf{Solution:} We have $\mu = np = 50 \times \frac{1}{100} = 0.5$.
    \begin{enumerate}[label=(\alph*)]
        \item $P(X \geq 1) = 1 - P(X = 0) = 1 - \frac{e^{-0.5}(0.5)^0}{0!} = 0.3935$.
        \item $P(X = 1) = \frac{e^{-0.5}(0.5)^1}{1!} = 0.3023$.
        \item $P(X \geq 2) = 1 - P(X \leq 1) = 1 - P(X = 0) - P(X = 1) = 0.09017$. \\
    \end{enumerate}
\end{example}


\begin{note}
    \phantom{}\
    \begin{enumerate}
        \item If $p$ is close to 1, we can still use Poisson to approximate Binomial, simply by interchanging the labels ``success'' and ``failure''. Now, we can get $P(\text{success})$ is close to 0.
    \end{enumerate}
\end{note}

\pagebreak

\subsection{Poisson Distribution from Poisson Process}

\begin{definition}[\textbf{Poisson Process}]
    \phantom{}  \\
    The following three conditions together define a \textbf{Poisson Process}:
    \begin{enumerate}
        \item \textbf{Independence:} the number of occurrences in non-overlapping intervals are independent.
        \item \textbf{Individuality:} $P(\text{2 or more events in $(t, t+\Delta t)$}) = o(\Delta t)$ (close to 0) as $\Delta t \to 0$.
        \item \textbf{Homogeneity or Uniformity:} Events occur at a homogeneous (uniform) rate $\lambda$ over time so that
        $P(\text{one event in $(t, t+\Delta t)$}) = \lambda \Delta t + o(\Delta t)$.
    \end{enumerate}
\end{definition}

In a Poisson process with rate of occurrence $\lambda$, the number of event occurrences $X$ in a time interval of length $t$
has a Poisson distribution with $\mu = \lambda t$.

\[f(x) = P(X=x) = \dis \frac{e^{-\lambda t} (\lambda t)^x}{x!} \text{, $x=0,1,2,\ldots$}\]


\textbf{Interpreting $\mu$ and $\lambda$.}
\begin{enumerate}
    \item $\lambda$ refers to the intensity or rate of occurrence. It represents the average rate of occurrence of events per unit of time.
    \item $\mu = \lambda t$ represents the average number of occurrences in $t$ units of time. \\
\end{enumerate}


\begin{example}
    Suppose earthquakes recorded in Ontario each year follow a Poisson process with an average of 6 per year.
    What is the probability that 7 will be recorded in a two year period?

    \textbf{Solution:} Let $X = \#$ of earthquakes in the two year period ($t = 2$). And the intensity of earthquakes is $\lambda = 6$ per year.
    So $\mu = \lambda t = 6 \times 2 = 12$ in two years. Then, $f(7) = \frac{e^{-12} 12^7}{7!} = 0.0437$. \\
\end{example}


The Poisson process also applies when ``events'' occur randomly in space. If $X$ represents the number of events in a volume or area in space of size $v$,
and if $\lambda$ is the average number of events per unit voilume (or area), then $X$ has a Poisson distribution with $\mu = \lambda v$.

The model is valid when we replace ``time'' by ``volume'' or ``area''.

\pagebreak

\begin{example}
    Coliform bacteria occur in a river water with an average intensity of 1 bacteria per 10 cubic centimeters (cc) of water. Find:
    \begin{enumerate}[label=(\alph*)]
        \item The probability there are no bacteria in a 20cc sample of water which is tested.
        
        \textbf{Solution:} Let $X = \#$ of Coliform bacteria observed in a specified volume. Then $X \sim \text{Poi}(\lambda = 1 \text{ per 10cc})$.
        We have $\mu = \lambda t = \frac{1}{10} \times 20 = 2$ per 20cc. So $P(X = 0) = \frac{e^{-2} 2^0}{0!} = e^{-2}$.
        \item The probability there are 5 or more bacteria in a 50cc sample.
        
        \textbf{Solution:} $\mu = \lambda t = \frac{1}{10} \times 50 = 5$ per 50cc. Then \vspace{-5mm}
        \begin{align*}
            P(X \geq 5) &= 1 - P(X \leq 4) \\
            &= 1 - \left[ P(X=0) + \cdots + P(X=4) \right] \\
            &= 1 - e^{-5} \left[ \frac{5^0}{0!} + \frac{5^1}{1!} + \cdots + \frac{5^4}{4!} \right].
        \end{align*}
    \end{enumerate}
\end{example}


\textbf{Distinguishing Poisson from Binomial and Other Distributions}

\begin{enumerate}
    \item Can we specify in advance the maximum value which $X$ can take? \\
    If we can, then the distribution is NOT Poisson. If there is no fixed upper limit, then might be Poisson.
    \item Does it make sense to ask how often the event did not occur? \\
    If it does make sense, the distribution is NOT Poisson. If it does not make sence, then might be Poisson.
\end{enumerate}


\subsection{Combining Other Models with the Poisson Process}

\begin{example}
    A very large (essentially infinite) number of ladybugs is released in a large orchard.
    They scatter randomly so that on average a tree has 6 ladybugs on it. Trees are all the same size.
    \begin{enumerate}[label=(\alph*)]
        \item Find the probability a tree has > 3 ladybugs on it. \\
        \textbf{Solution:} Poisson distribution with $\lambda = 6$ and $v = 1$ (that is, any tree has a ``volume'' of one unit). So $\mu = \lambda v = 6$.
        Then $P(X > 3) = 1 - P(X \leq 3) = 1 - e^{-6}\left[ \frac{6^0}{0!} + \frac{6^1}{1!} + \frac{6^2}{2!} + \frac{6^3}{3!} \right] = 0.8488$.
        \item When 10 trees are picked at random, what is the probability that 8 of these trees have > 3 ladybugs on them? \\
        \textbf{Solution:} Binomial distribution where success means >3 ladybugs on a tree. We have $X \sim \text{Bin}(10, 0.8488)$.
        Then $P(X = 8) = \binom{10}{8}(0.8488)^8 (1-0.8488)^2 = 0.2772$.
        \item Trees are checked until 5 with > 3 ladybugs are found. Let $X$ be the total number of trees checked. Find the probability function, $f(x)$. \\
        \textbf{Solution:} $X-5 \sim \text{NB}(5, 0.8488)$ with $(x-5)$ failures. Then
        \begin{align*}
            f(x) &= \binom{x-5+5-1}{x-5}(0.8488)^5 (1-0.8488)^{x-5} \\
            &= \binom{x-1}{x-5}(0.8488)^5 (0.1512)^{x-5} \\
            &= \binom{x-1}{4}(0.8488)^5 (0.1512)^{x-5} \text{, $x = 5,6,7,\ldots$}
        \end{align*}
        \item Find the probability a tree with > 3 ladybugs on it has exactly 6. \\
        \textbf{Solution:} Let $A = {\text{6 ladybugs}}$ and $B = {\text{more than 3 ladybugs}}$. \\
        Then, $P(A|B) = \frac{P(A \cap B)}{P(B)} = \frac{P(A)}{P(B)} = \frac{\frac{e^{-6} 6^6}{6!}}{0.8488} = 0.1892$.
        \item On 2 trees there are a total of $t$ ladybugs. Find the probability that $x$ of these are on the first tree. \\
        \textbf{Solution:}
        \begin{align*}
            P(\text{$x$ on first tree} | \text{total of $t$}) &= \frac{P(\text{$x$ on first tree} \cap \text{total of $t$})}{P(\text{total of $t$})} \\
            &= \frac{P(\text{$x$ on first tree} \cap \text{$t-x$ on second tree})}{P(\text{total of $t$})} \\
            &= \frac{P(\text{$x$ on first tree}) \cdot P(\text{$t-x$ on second tree})}{P(\text{total of $t$})}
        \end{align*}
        Use Poisson distribution with $\mu = 6 \times 2 = 12$ in the denominator since there are two trees.
        \begin{align*}
            &= \frac{ \left( \frac{e^{-6} 6^x}{x!} \right) \left( \frac{e^{-6} 6^{t-x}}{(t-x)!} \right) }{\frac{e^{-12} 12^t}{t!}} \\
            &= \frac{t!}{x!(t-x)!} \left( \frac{6}{12} \right)^x \left( \frac{6}{12} \right)^{t-x} \\
            &= \binom{t}{x} \left( \frac{1}{2} \right)^x \left( 1 - \frac{1}{2} \right)^{t-x} \text{ for $x = 0,1,\ldots, t$}. 
        \end{align*}
        \begin{remark}
            We can also let $X = \#$ of ladybugs on the first tree, then $X \sim \text{Bin}(t, 0.5)$. Since the probability of a ladybug on the first tree is just 0.5 as there are two trees.
        \end{remark}
    \end{enumerate}
\end{example}


\subsection{Summary of Probability Functions for Discrete Random Variables}

\begin{center}
    \begin{tabular}{l l}
        \textbf{Name} & \textbf{Probability Function} \\ \hline \addlinespace \addlinespace
        Discrete Uniform & $\dis f(x) = \frac{1}{b - a + 1} \quad \text{for } x = a, a + 1, a + 2, \ldots, b \quad (b > a)$ \vspace{10pt}    \\
        Hypergeometric & $\dis f(x) = \frac{\binom{r}{x} \binom{N-r}{n-x}}{\binom{N}{n}} \quad \text{for } x = \max(0, n - (N - r)), \ldots, \min(n, r)$ \vspace{10pt} \\
        Binomial & $\dis f(x) = \binom{n}{x} p^x(1 - p)^{n-x} \quad \text{for } x = 0, 1, 2, \ldots, n \quad (0 < p < 1)$ \vspace{10pt}  \\
        Negative Binomial & $\dis f(x) = \binom{x+k-1}{x} p^k(1 - p)^x \quad \text{for } x = 0, 1, 2, \ldots \quad (0 < p < 1)$ \vspace{10pt}    \\
        Geometric & $\dis f(x) = p(1 - p)^x \quad \text{for } x = 0, 1, 2, \ldots \quad (0 < p < 1)$ \vspace{10pt}   \\
        Poisson & $\dis f(x) = \frac{e^{-\mu}\mu^x}{x!} \quad \text{for } x = 0, 1, 2, \ldots \quad (\mu > 0)$ \vspace{30pt}  \\
    \end{tabular}
\end{center}

% An example of drawing probability function table:   \\
% \begin{tabular}{l|*{5}{c}}
%     $x$ & 1 & 2 & 3 & 4 & 5 \\
%     \hline
%     $F(x)$ & $0.1k$ & $0.2$ & $0.5k$ & $k$ & $4k^2$ \\
% \end{tabular}



\newpage