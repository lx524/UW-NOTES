\subsection{Definitions of Probability}

\begin{definition}[\textbf{Experienment}]
    An \textbf{experiment} is a situation involving chance or uncertainty that leads to results called outcomes.
\end{definition}

\begin{definition}[\textbf{Outcome}]
    An \textbf{outcome} is the result of a single trial (attempt) of an experiment.
\end{definition}

\begin{definition}[\textbf{Event}]
    An \textbf{event} is one or more outcomes of an experiment.
\end{definition}

\begin{definition}[\textbf{Sample Space, $S$}]
    The set of ALL possible distinct outcomes in a random experiment is called the \textbf{sample space}, $S$.    
\end{definition}

\begin{definition}[\textbf{Probability}]
    \phantom{}
    \begin{enumerate}
        \item \textbf{Classical} definition: 
        \[P(\text{event}) = \frac{\text{\# of ways the event can occur}}{\text{\# of all possible outcomes}}\]
        provided all outcomes are equally likely.
        \item \textbf{Relative frequency} definition:
        \[P(\text{event}) = \text{proportion of times the event occurs in a long series of repeated experiment}\]
        \item \textbf{Subjective probability} definition:
        \[P(\text{event}) = \text{how certain we are that the event will occur}\]
    \end{enumerate}
\end{definition}

\begin{note}
    All three definitions have serious limitations.
\end{note}

\begin{example}[\textbf{Rolling a die}]
    $S = \{1, 2, \ldots, 6\}$. If a die is rolled once, the number 2 can  be observed in exactly 1 out of 6 ways.
\end{example}

% \newpage

% A latex envrioment reference page, to be deleted at the end of the term...
% \begin{claim}
%     Hi, $\vecspan{\vec{\{x\}}} \Iff$
% \end{claim}

% \begin{theorem}
%     \contradiction \st $\expect{X}$ and $\Var{X}$
% \end{theorem}

% \begin{proof}
%     Follows immediately lol.
% \end{proof}

% \begin{proposition}
%     hiiiii  $\eval{\frac{\dd{y}}{\dd{x}}}_{x_n}$
% \end{proposition}

% \begin{lemma}
%     bonjour
% \end{lemma}

% \begin{corollary}
%     nihao
% \end{corollary}

% \begin{example}
%     simimasai
% \end{example}

% \begin{remark}
%     666
% \end{remark}

% \begin{note}
%     iwhsfn 
% \end{note}

% \begin{definition}
    
% \end{definition}

% An example of drawing probability function table:   \\
% \begin{tabular}{l|*{5}{c}}
%     $x$ & 1 & 2 & 3 & 4 & 5 \\
%     \hline
%     $F(x)$ & $0.1k$ & $0.2$ & $0.5k$ & $k$ & $4k^2$ \\
% \end{tabular}

% \begin{tikzpicture}[samples=100,scale=1.4]
%     \draw[-Stealth](-3.2,0)--(0,0)node[below right]{$O$}--(3.2,0)node[below]{$x$};
%     \draw[-Stealth](0,-2)--(0,2)node[left]{$y$};
%     \draw[domain=-pi:pi]plot(\x,{sin(\x r)});
%     \draw[semithick,domain=-pi/2:pi/2]plot({sin(\x r)},\x);
%     \node at(1.9,1.2){$y=\sin x$};\node at(1.3,1.7){$y=\arcsin x$};
%     \draw[densely dashed](1,pi/2)--(1,0)node[below]{$1$}
%     (1,pi/2)--(0,pi/2)node[left]{$\frac\pi2$};
%     \draw[densely dashed](-1,-pi/2)--(-1,0)node[above]{$-1$}
%     (-1,-pi/2)--(0,-pi/2)node[right]{$-\frac\pi2$};
% \end{tikzpicture}



\newpage